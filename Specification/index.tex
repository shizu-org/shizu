\documentclass{article}
\usepackage{amsmath}
\usepackage{amsfonts}

\newcommand{\myproject}{Shizu}
\newcommand{\myauthor}{Michael Heilmann (michael.heilmann@shizu.org)}

\title{\myproject{} Abstract Machine Specification}
\author{\myauthor}


\begin{document}
\maketitle
\section{Abstract}
This document describes the \emph{\myproject{} Abstract Machine} which executes programs written in the \emph{\myproject
{} Programming Language}. A concrete machine implementation is considered as conforming with the \emph{\myproject{}
Abstract Machine} if it fulfils all mandatory properties of the \emph{\myproject{} Abstract Machine} described here. To
describe the \emph{\myproject{} Abstract Machine} this document uses operational semantics: It starts out with a
description of the initial state of the \emph{\myproject{} Abstract Machine} and then proceeds to describe how its
execution of \emph{\myproject{} Programming Language} constructs mutates that state.

\section{The \myproject Programming Language}

\subsection{Types}
The \myproject{} Programming Language is a dynamically typed language. This means that variables do not have types and
only values do have types. All values carry their own type. All values PL are first-class values. This means that
all values can be stored in variables, passed as arguments to functions and operators, and can be returned by functions
and operators.

The \myproject{} Programming Language knows the types $\textit{Boolean}$, $\textit{Integer}$, $\textit{Type}$, and
$\textit{Void}$.The type $\textit{Boolean}$ has two values $\textit{true}$ and $\textit{false}$. The type
$\textit{Integer}$ has 64 bit two's complement integer values. Implementations may allow for 32 bit two's complement
integer values. The type $\textit{Type}$ has values uniquely identifiying and describing types.

\subsection{Registers}
Values of these types can be store in and loaded from locations called registers. Each register is assigned an unique
number from a consecutive sequence of natural numbers starting at $0$ and ending at $n-1$ where $n \geq 32$ is the
maximum number of registers of an implementation. Registers in the \myproject{} Programming Language have names of the
form $\textit{\$[0-9]+}$. Note that leading zeroes are ignored, hence $\$0$ and $\$00$ both identify the same register
with the register number $0$.

\subsection{Types of values}
You can obtain the type of any value using the $\texttt{typeof}$ operator for example $\texttt{\$1 = typeof(\$0)}$
determines the type of the value stored in register $\texttt{\$0}$ and stores it in register $\texttt{\$1}$.

\end{document}
