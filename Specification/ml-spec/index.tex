\documentclass{article}
\usepackage{amsmath}
\usepackage{amsfonts}

\newcommand{\myproject}{Shizu}
\newcommand{\myauthor}{Michael Heilmann (michael.heilmann@shizu.org)}
\newcommand{\mymachine}{\myproject{} Machine}
\newcommand{\mylanguage}{\myproject{} \mymachine{} Language}

\title{\mylanguage{}}
\author{\myauthor}


\begin{document}
\maketitle
\section{Abstract}
This document describes the \emph{\mylanguage{}}.
The \emph{\mylanguage{}} is designed as a target language that is executed by a suitable interpreter.
An interpreter is called a \emph{\mymachine{}} if it fulfils all properties of the \emph{\mylanguage{}} described in this document. 

%\section{\mylanguage}

\section{Introduction}
The \myproject{} Programming Language is a dynamically typed language. This means that variables do not have types and
only values do have types. All values carry their own type. All values PL are first-class values. This means that
all values can be stored in variables, passed as arguments to functions and operators, and can be returned by functions
and operators.

The \myproject{} Programming Language knows the types $\textit{Boolean}$, $\textit{Integer}$, $\textit{Real}$,
$\textit{String}$, $\textit{Type}$, and $\textit{Void}$.The type $\textit{Boolean}$ has two values $\textit{true}$
and $\textit{false}$. The type $\textit{Integer}$ has 64 bit two's complement integer values. Implementations may
allow for 32 bit two's complement integer values. Values of type $\textit{String}$ are UTF-8 encoded Unicode strings.
The type $\textit{Type}$ has values uniquely identifiying and describing types.

\noindent\fbox{\parbox{\textwidth}{%
Future revisions of this document will add support for the types $\textit{List}$ and $\textit{Map}$.
}}

\section{Lexical Structure}
\noindent{}The lexical analysis of a \emph{\mylanguage{}} program can be described as a three stage lexical analysis.

\subsection{Stage 1}
\noindent{}The 1st stage of the lexical analysis analyses the input Bits into a sequence of Unicode input symbols
$\textit{unicodeInputSymbol}$. This sequence is prefixed the special input symbol $\textit{startOfInput}$ and suffixed with
the special input symbol $\textit{endOfInput}$. Note that $\textit{startOfInput}$ and $\textit{endOfInput}$ both must
be different from any input symbol $\textit{unicodeInputSymbol}$.

\noindent{}That is, the 1st stage of the lexical analysis analyses a program $\pi$ as
\begin{flalign*}
\pi_1 : \textit{startOfInput}\;\textit{unicodeInputSymbol}^* \;\textit{endOfInput}
\end{flalign*}

\subsection{Stage 2}
\noindent{}The 2nd stage of the lexical analysis analyses the sequence of input symbols
after the $\textit{startOfInput}$ special input symbol into a sequence of input  lines.
An input line is a sequence of zero or more input symbols without $\textit{return}$  or
$\textit{newline}$ or $\textit{startOfInput}$ or $\textit{endOfInput}$. An input   line
is terminated by either a line terminator $\textit{lineTerminators}$ or the end of  the
special input symbol $\textit{endOfInput}$.

\begin{flalign*}
\textit{line}          \;:\;& \textit{inputSymbol}^*\\ 
\textit{lineTerminator}\;:\;& \textit{newline}\\
                            & \textit{return}\\ 
                            & \textit{return}\;\textit{newline}\\
                            & \textit{return}\;\textit{newline}\\
\textit{inputSymbol}   \;:\;&\textit{unicodeInputSymbol} - (\textit{return}\vert\textit{newline})
\end{flalign*}

\noindent{}That is, a program $\pi$ is of the form
\begin{flalign*}
\pi_2 : \textit{startOfInput} (\textit{inputLines}\;\textit{lineTerminator})^* \;\textit{inputLines}\;\textit{endOfInput}
\end{flalign*}

\subsection{Stage 3}
\noindent{}The 3rd stage of the lexical analysis analyses each $\textit{inputLines}$ into a sequence of
input elements $\textit{inputElement}$.

\begin{flalign*}
\pi_3 : \textit{startOfInput}\;\textit{inputElement}^* \;\textit{endOfInput}
\end{flalign*}


\noindent{}An input element is a token $\textit{token}$ or a comment $\textit{comment}$:
\begin{flalign*}
\textit{inputElement}\;:\;& \textit{comment}\\
                          & \textit{token}
\end{flalign*}

 
\noindent{}A comment $\textit{comment}$ is a single-line comment $\textit{singleLineComment}$ or a $\textit{multiLinecomment}$:
\begin{flalign*}
\textit{comment}     \;:\;& \textit{singleLineComment}\\
                          & \textit{multiLineComment}
\end{flalign*}


\noindent{}A token $\textit{token}$ is a $\textit{literal}$, an $\textit{operator}$, or a $\textit{name}$:
\begin{flalign*}
\textit{token}     \;:\;& \textit{literal}\\
                        & \textit{operator}\\
                        & \textit{name}\\
                        & \textit{whitespace}
\end{flalign*}

\subsubsection{Names}
\noindent{}A name $\textit{name}$ is a single alphabetic character or underscore followed
by zero or more alphabetic characters, digits, or underscores:
\begin{flalign*}
\textit{name}       \;:\;&(\textit{alphabetic} \;\texttt{'\_'}) (\textit{alphabetic}\;\texttt{'\_'}\;\textit{digit})^*\\
\textit{alphabetic} \;:\;&\texttt{'A'}-\texttt{'Z'}\\
                         &\texttt{'a'}-\texttt{'z'}\\
\textit{digit}\;:\;      &\texttt{'0'}-\texttt{'9'}
\end{flalign*}

\subsubsection{Literals}
\noindent{}A literal $\textit{literal}$ is either a $\textit{booleanLiteral}$, $\textit{integerLiteral}$, $\textit{stringLiteral}$, or $\textit{floatingPointLiteral}$.
\begin{flalign*}
\textit{literal}   \;:\;& \textit{booleanLiteral}\\
                        & \textit{floatingPointLiteral}\\
                        & \textit{integerLiteral}\\
                        & \textit{stringLiteral}
\end{flalign*}


\noindent{}A boolean literal $\textit{booleanLiteral}$ is $\texttt{'true'}$ and $\texttt{'false'}$.
\begin{flalign*}
\textit{booleanLiteral}   \;:\;& \texttt{'true'}\\
                               & \texttt{'false'}
\end{flalign*}
A $\textit{booleanLiteral}$ is always of the type $\textit{Boolean}$.\newline


\noindent{}An $\textit{integerLiteral}$ is begins with an optional sign and is followed by one or more decimal digits.
\begin{flalign*}
\textit{integerLiteral}   \;:\;& (\texttt{'+'} \vert \texttt{'-'})^? \; \textit{digit}^+
\end{flalign*}
A $\textit{integerLiteral}$ is always of the type $\textit{Integer}$.\newline


\noindent{}A $\textit{stringLiteral}$ opens with a double quote followed by zero or more string characters or escape sequences
closed by a double quote. A string character is any characters within the range $\texttt{0x20}$ (inclusive) to $\texttt{0x7E}$
(inclusive) excluding the double quote and the backslash.                    Escape sequences are $\texttt{\textbackslash n}$,
$\texttt{\textbackslash r}$, $\texttt{\textbackslash t}$, $\texttt{\textbackslash \textbackslash}$, and $\texttt{\textbackslash "}$.

\begin{flalign*}
\textit{stringLiteral}        \;:\;& \texttt{'"'} \textit{stringCharacter}^* \texttt{'"'}\\
\textit{stringCharacter}      \;:\;& (\textit{inputCharacter} \vert \textit{stringEscapeSequence}) - (\texttt{'"'} \vert \texttt{'\textbackslash'})\\
\textit{stringEscapeSequence} \;:\;& \texttt{'\textbackslash n'} \vert \texttt{'\textbackslash r'} \vert \texttt{'\textbackslash t'} \vert \texttt{'\textbackslash \textbackslash'} \vert \texttt{'\textbackslash "'}
\end{flalign*}

\noindent\fbox{\parbox{\textwidth}{%
Future revisions of this document will provide improved Unicode support.
The first measure to improve Unicode support is extending the set of Unicode symbols that are allowed as string characters.
The second measure to improve Unicode support is the addition of Unicode escape sequences $\texttt{\textbackslash +Uxxxx}$.
}}

\noindent{}A $\textit{floatingPointLiteral}$ always begins with an optional sign.
It then of one of the two following forms:
\begin{enumerate}
  \item a period followed by one or more decimal digits
  \item one or more decimal digits followed by a period
\end{enumerate}
It ends with an optional exponent.

\begin{flalign*}
\textit{floatingPointLiteral} \;:\;& (\texttt{'+'}\vert\texttt{'-'})^? \; [(\textit{digit}^+ \;\texttt{'.'} \;\textit{digit}*) \vert (\texttt{'.'} \;\textit{digit}+)] \;\textit{floatingPointExponent}?
\end{flalign*}

\noindent{}An exponent begins with an exponent character followed by an optional sign followed by one or more decimal digits.
\begin{flalign*}
\textit{floatingPointExponent} \;:\;& (\texttt{'e'}\vert\texttt{'E'}) \; (\texttt{'+'} \vert \texttt{'-'})^? \; \textit{digit}^+
\end{flalign*}

\subsubsection{Comments}
\noindent{}A comment $\textit{comment}$ is a single-line comment $\textit{singleLineComment}$ or a multi-line comment $\textit{multiLineComment}$.
\begin{flalign*}
\textit{comment}  \;:\;& \textit{singleLineComment}\\
                       & \textit{multiLineComment}
\end{flalign*}

\noindent{}A single-line comment $\textit{singleLineComment}$ begins with two consecutive slashes $\textit{'//'}$ and extends until the end of its input line line.
\begin{flalign*}
\textit{singleLineComment}  \;:\;& \texttt{'//'} \textit{inputSymbol}^*
\end{flalign*}

\noindent{}A multi-line comment $\textit{multiLineComment}$ begins with a slash followed by a star.
It ends with a slash followed by a star.

\begin{flalign*}
\textit{multiLineComment}     \;:\;& \texttt{'/*'} \textit{multiLineCommentTail}\\
\textit{multiLineCommentTail} \;:\;& \texttt{'*'} \textit{multiLineCommentTailStar}\\
                                   & \textit{NotStar} \textit{multiLineCommenTail}\\
\textit{multiLineCommentTailStar} \;:\;& \texttt{'/'} \\
                                       & \texttt{'*'} \textit{multiLineCommentTailStar}\\
                                       & \textit{NotStarNotSlash} \textit{multiLineCommenTail}\\
\textit{NotStar}          \;:\;&\textit{inputSymbol} - \texttt{'*'}\\
                               &\textit{lineTerminator}\\
\textit{NotStarNotSlash} \;:\;&\textit{inputSymbol} - (\texttt{'*'}\vert\texttt{'/'})\\
                              &\textit{lineTerminator}                    
\end{flalign*}


\subsection{Registers}
Values of these types can be store in and loaded from locations called registers. Each register is assigned an unique
number from a consecutive sequence of natural numbers starting at $0$ and ending at $n-1$ where $n \geq 32$ is the
maximum number of registers of an implementation. Registers in the \myproject{} Programming Language have names of the
form $\textit{\$[0-9]+}$. Note that leading zeroes are ignored, hence $\$0$ and $\$00$ both identify the same register
with the register number $0$. The initial value of any register is the $\texttt{void}$ value.

\subsection{Types and Values}
You can obtain the type of any value using the $\texttt{typeof}$ operator for example $\texttt{\$1 = typeof(\$0)}$
determines the type of the value stored in register $\texttt{\$0}$ and stores it in register $\texttt{\$1}$.

\end{document}
