\documentclass{article}
\usepackage{amsmath}
\usepackage{amsfonts}

\newcommand{\myproject}{Shizu}
\newcommand{\myauthor}{Michael Heilmann (michael.heilmann@shizu.org)}
\newcommand{\mymachine}{\myproject{} Machine}
\newcommand{\mylanguage}{\myproject{} \mymachine{} Language}

\title{\mylanguage{}}
\author{\myauthor}


\begin{document}
\maketitle
\section{Abstract}
This document describes the \emph{\mylanguage{}}.
The \emph{\mylanguage{}} is designed as a target language that is executed by a suitable interpreter.
An interpreter is called a \emph{\mymachine{}} if it fulfils all properties of the \emph{\mylanguage{}} described in this document. 

%\section{\mylanguage}

\section{Introduction}
The \myproject{} Programming Language is a dynamically typed language. This means that variables do not have types and
only values do have types. All values carry their own type. All values PL are first-class values. This means that
all values can be stored in variables, passed as arguments to functions and operators, and can be returned by functions
and operators.

The \myproject{} Programming Language knows the types $\textit{Boolean}$, $\textit{Integer}$, $\textit{Real}$,
$\textit{String}$, $\textit{Type}$, and $\textit{Void}$.The type $\textit{Boolean}$ has two values $\textit{true}$
and $\textit{false}$. The type $\textit{Integer}$ has 64 bit two's complement integer values. Implementations may
allow for 32 bit two's complement integer values. Values of type $\textit{String}$ are UTF-8 encoded Unicode strings.
The type $\textit{Type}$ has values uniquely identifiying and describing types.

\noindent\fbox{\parbox{\textwidth}{%
Future revisions of this document will add support for the types $\textit{List}$ and $\textit{Map}$.
}}

\section{Lexical Conventions}
\emph{\mylanguage{}} is a free-form language. It ignores spaces and comments between lexical elements (tokens),
except as delimiters between two tokens. In source code, \emph{\mylanguage{}} recognizes as spaces the standard
ASCII whitespace characters space, form feed, newline, carriage return, horizontal tab, and vertical tab. 

The values $\textit{true}$ and $\textit{false}$ are represented in the language by
the literal $\texttt{true}$ and $\texttt{false}$, respectively. The values of type
$\textit{String}$ are represented in the language by a double quote followed by zero
or more string characters or escape sequences followed by a double quote. 
A string character is any characters within the range $\texttt{0x20}$ (inclusive) to $\texttt{0x7E}$
(inclusive) excluding the double quote and the backslash. Escape sequences are $\texttt{\textbackslash n}$,
$\texttt{\textbackslash r}$, $\texttt{\textbackslash t}$, $\texttt{\textbackslash}$, and $\texttt{\textbackslash "}$.

\noindent\fbox{\parbox{\textwidth}{%
Future revisions of this document will provide improved Unicode support.
The first measure to improve Unicode support is extending the set of Unicode symbols that are allowed as string characters.
The second measure to improve Unicode support is the addition of Unicode escape sequences $\texttt{\textbackslash +Uxxxx}$.
}}

\subsection{Registers}
Values of these types can be store in and loaded from locations called registers. Each register is assigned an unique
number from a consecutive sequence of natural numbers starting at $0$ and ending at $n-1$ where $n \geq 32$ is the
maximum number of registers of an implementation. Registers in the \myproject{} Programming Language have names of the
form $\textit{\$[0-9]+}$. Note that leading zeroes are ignored, hence $\$0$ and $\$00$ both identify the same register
with the register number $0$.

\subsection{Types of values}
You can obtain the type of any value using the $\texttt{typeof}$ operator for example $\texttt{\$1 = typeof(\$0)}$
determines the type of the value stored in register $\texttt{\$0}$ and stores it in register $\texttt{\$1}$.

\end{document}
